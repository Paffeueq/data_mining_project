\documentclass[12pt,a4paper]{article}
\usepackage[utf8]{inputenc}
\usepackage[polish]{babel}
\usepackage[T1]{fontenc}
\usepackage{graphicx}
\usepackage{hyperref}
\usepackage{geometry}
\usepackage{enumitem}
\usepackage{booktabs}
\usepackage{longtable}
\usepackage{fancyhdr}
\usepackage{lastpage}
\usepackage{float}

\geometry{margin=2.5cm}

\hypersetup{
	colorlinks=true,
	linkcolor=blue,
	filecolor=magenta,      
	urlcolor=cyan,
}

\title{Analiza ekspolracyjna}
\author{Grupa 6}

\date{\today}

\makeatletter

\begin{document}
	
	\pagestyle{fancy}
	\fancyhead{}
	\fancyhead[L]{\@title}
	\fancyhead[R]{\@author}
	\fancyhead[C]{\@date}
	
	\cfoot{\thepage\ / \pageref{LastPage}}
	
	\newpage
\begin{center}
	\vspace*{2cm}
	{\Huge \textbf{ANALIZA EKSPLORACYJNA DANYCH}}\\[1cm]
	{\LARGE Oczekiwana długość życia - Dane WHO}\\[2cm]
	
	{\Large \textbf{Projekt: Metody Eksploracji Danych}}\\[1.5cm]
	
	{\large \textbf{Autorzy:}}\\[0.5cm]
	{\large Paweł Myszka}\\
	{\large Stanisław Dutkiewicz}\\
	{\large Łukasz Jugo}\\[2cm]
	
	{\large Grudzień 2025}
	\vspace*{3cm}
\end{center}
	
	\newpage
	
	\tableofcontents
	\newpage
	
\section{Opis danych w strukturze}

\subsection{Wymiary zbioru danych}

Zbiór danych WHO dotyczący oczekiwanej długości życia zawiera:
\begin{itemize}
	\item \textbf{Liczba wierszy:} 2938 obserwacji
	\item \textbf{Liczba kolumn:} 22 zmienne
	\item \textbf{Zmienne numeryczne:} 20
	\item \textbf{Zmienne kategorialne:} 2 (Country, Status)
\end{itemize}

\subsection{Zmienne w zbiorze danych}

\begin{table}[h]
	\centering
	\begin{tabular}{|l|l|l|}
		\hline
		\textbf{Lp.} & \textbf{Nazwa zmiennej} & \textbf{Typ danych} \\
		\hline
		1 & Country & object \\
		2 & Year & int64 \\
		3 & Status & object \\
		4 & Life expectancy & float64 \\
		5 & Adult Mortality & float64 \\
		6 & infant deaths & int64 \\
		7 & Alcohol & float64 \\
		8 & percentage expenditure & float64 \\
		9 & Hepatitis B & float64 \\
		10 & Measles & int64 \\
		11 & BMI & float64 \\
		12 & under-five deaths & int64 \\
		13 & Polio & float64 \\
		14 & Total expenditure & float64 \\
		15 & Diphtheria & float64 \\
		16 & HIV/AIDS & float64 \\
		17 & GDP & float64 \\
		18 & Population & float64 \\
		19 & thinness 1-19 years & float64 \\
		20 & thinness 5-9 years & float64 \\
		21 & Income composition of resources & float64 \\
		22 & Schooling & float64 \\
		\hline
	\end{tabular}
	\caption{Zmienne w zbiorze danych WHO Life Expectancy}
	\label{tab:variables}
\end{table}

\subsection{Zmienna docelowa}

\textbf{Life expectancy} (Oczekiwana długość życia w latach) jest zmienną objaśnianą w analizie. 
Pozostałe 19 zmiennych numerycznych pełnią rolę zmiennych objaśniających.

\subsection{Rozkład brakujących wartości}

\begin{table}[h]
	\centering
	\begin{tabular}{|l|c|c|}
		\hline
		\textbf{Zmienna} & \textbf{Non-Null Count} & \textbf{\% Kompletności} \\
		\hline
		Country & 2938 & 100.0\% \\
		Year & 2938 & 100.0\% \\
		Status & 2938 & 100.0\% \\
		infant deaths & 2938 & 100.0\% \\
		Measles & 2938 & 100.0\% \\
		percentage expenditure & 2938 & 100.0\% \\
		under-five deaths & 2938 & 100.0\% \\
		HIV/AIDS & 2938 & 100.0\% \\
		\hline
		Life expectancy & 2928 & 99.7\% \\
		Adult Mortality & 2928 & 99.7\% \\
		Alcohol & 2744 & 93.4\% \\
		Hepatitis B & 2385 & 81.2\% \\
		BMI & 2904 & 98.8\% \\
		GDP & 2490 & 84.8\% \\
		Population & 2286 & 77.8\% \\
		Total expenditure & 2712 & 92.3\% \\
		\hline
	\end{tabular}
	\caption{Kompletność danych dla głównych zmiennych}
	\label{tab:completeness}
\end{table}

\noindent
Tak przygotowane dane wymagają obróbki braków wartości przed modelowaniem.
	
	\subsection{Analiza}
	Zbiór obejmuje dane z 161 krajów za lata 2000-2015. Struktura jest regularna - każda obserwacja reprezentuje kraj-rok. Zmienne obejmują wskaźniki zdrowotne (śmiertelność, choroby zakaźne, HIV/AIDS), ekonomiczne (PKB, ekspozycja zdrowotna, skład dochodów) i społeczne (szkolnictwo, alkohol).
	
\section{Analiza zależności - korelacje zmiennych}

\subsection{Wprowadzenie}
W ramach tej analizy obliczono dwie macierze korelacji Pearsona:

\begin{enumerate}
	\item \textbf{Macierz korelacji wszystkich zmiennych} --- Macierz o wymiarach $(20 \times 20)$ prezentująca powiązania między wszystkimi 20 zmiennymi numerycznymi.
	
	\item \textbf{Macierz korelacji zmiennej docelowej ze zmiennymi objaśniającymi} --- Wektor korelacji $(20 \times 1)$ pokazujący siłę i kierunek powiązań każdej zmiennej ze zmienną docelową (Life expectancy). Macierz ta jest podstawą do identyfikacji zmiennych o największym wpływie na oczekiwaną długość życia.
\end{enumerate}

\begin{figure}[H]
	\centering
	\includegraphics[width=0.9\textwidth]{1.png}
	\caption{Macierz korelacji zmiennych i korelacje ze zmienną docelową Life expectancy}
	\label{fig:macierz_korelacji}
\end{figure}

\subsection{Top 10 predyktorów}

Tabela \ref{tab:top10_korelacje} prezentuje dziesięć zmiennych o największej bezwzględnej wartości korelacji ze zmienną docelową, uporządkowanych malejąco.

\begin{table}[h]
	\centering
	\caption{Top 10 predyktorów zmiennej Life expectancy wg siły korelacji}
	\label{tab:top10_korelacje}
	\begin{tabular}{|l|c|c|c|}
		\hline
		\textbf{Zmienna} & \textbf{Korelacja (r)} & \textbf{Siła} & \textbf{Kierunek} \\
		\hline
		Schooling & +0,752 & SILNA & dodatnia \\
		Income composition of resources & +0,725 & SILNA & dodatnia \\
		Adult Mortality & -0,696 & UMIARKOWANA & ujemna \\
		BMI & +0,568 & UMIARKOWANA & dodatnia \\
		HIV/AIDS & -0,557 & UMIARKOWANA & ujemna \\
		Diphtheria & +0,479 & UMIARKOWANA & dodatnia \\
		thinness 1-19 years & -0,477 & UMIARKOWANA & ujemna \\
		thinness 5-9 years & -0,472 & UMIARKOWANA & ujemna \\
		Polio & +0,466 & UMIARKOWANA & dodatnia \\
		GDP & +0,461 & UMIARKOWANA & dodatnia \\
		\hline
	\end{tabular}
\end{table}

\subsection{Interpretacja wyników}

\subsubsection{Zmienne o silnej korelacji}
Dwie zmienne wykazują silne powiązanie ze zmienną docelową ($|r| > 0,70$):

\begin{itemize}
	\item \textbf{Schooling} ($r = +0,752$) --- Lata edukacji przypadające na mieszkańca wykazują najsilniejszą dodatnią korelację z oczekiwaną długością życia. Wyższe poziomy wykształcenia społeczeństwa są powiązane ze zwiększoną średnią długością życia, co sugeruje, że edukacja jest jednym z najważniejszych determinantów zdrowia populacji.
	
	\item \textbf{Income composition of resources} ($r = +0,725$) --- Skład dochodów kraju (proporcja dochodu wydawanego na jedzenie, mieszkanie itp.) wykazuje drugie miejsce pod względem siły korelacji. Lepsze zasoby ekonomiczne i ich właściwy przydział są istotnie powiązane z wyższą długością życia.
\end{itemize}

\subsubsection{Zmienne o umiarkowanej korelacji}
Osiem zmiennych wykazuje umiarkowaną korelację ($0,40 < |r| < 0,70$) ze zmienną docelową:

\begin{itemize}
	\item \textbf{Adult Mortality} ($r = -0,696$) --- Ujemna korelacja wskazuje, że wyższa śmiertelność dorosłych jest związana z niższą oczekiwaną długością życia.
	
	\item \textbf{BMI} ($r = +0,568$) --- Średni indeks masy ciała populacji wykazuje dodatnią korelację, sugerując, że populacje z wyższym średnim BMI mają dłuższą średnią długość życia.
	
	\item \textbf{HIV/AIDS} ($r = -0,557$) --- Ujemna korelacja odzwierciedla fakt, że wyższe rozpowszechnienie HIV/AIDS negatywnie wpływa na oczekiwaną długość życia.
	
	\item \textbf{Wskaźniki wyszczepienia} (Diphtheria, Polio) --- Dodatnie korelacje wskaźników wyszczepienia ($r \approx +0,47/+0,48$) odzwierciedlają dostęp do opieki medycznej i profilaktyki zdrowotnej.
	
	\item \textbf{Niedowaga wśród dzieci} (thinness 1-19, thinness 5-9 years) --- Obie zmienne wykazują ujemne korelacje ($r \approx -0,47/-0,48$), wskazując na związek niedowagi z niższą oczekiwaną długością życia.
	
	\item \textbf{GDP} ($r = +0,461$) --- Produkt krajowy brutto kraju wykazuje dodatnią korelację, co sugeruje, że bogatsza gospodarki osiągają wyższe średnie długości życia.
\end{itemize}

\subsection{Podsumowanie}
Przeprowadzona analiza korelacji ujawnia, że najważniejszymi determinantami oczekiwanej długości życia są:

\begin{enumerate}
	\item \textbf{Czynniki społeczno-ekonomiczne} --- edukacja i dochody populacji wykazują najsilniejsze powiązania ze zmienną docelową
	\item \textbf{Czynniki zdrowotne} --- dostęp do opieki medycznej (wyszczepienia), rozpowszechnienie chorób zakaźnych (HIV/AIDS), oraz wskaźniki zdrowia populacji
	\item \textbf{Zasobność ekonomiczna} --- PKB kraju wpływa istotnie na możliwość zapewnienia opieki zdrowotnej
\end{enumerate}
	
	\section{Zakresy i stopień zmienności zmiennych}
	
	W ramach tego etapu obliczono podstawowe statystyki opisowe dla każdej zmiennej, w tym miary tendencji centralnej (średnia, mediana) oraz miary rozproszenia (odchylenie standardowe, współczynnik zmienności). Współczynnik zmienności (CV\%) wyrażony wzorem:
	
	$$CV\% = \frac{\sigma}{\mu} \times 100$$
	
	gdzie $\sigma$ oznacza odchylenie standardowe, a $\mu$ średnią, pozwala na porównanie zmienności między zmiennymi o różnych skalach pomiarowych.
	
	\subsection{Zmienne o ekstremalnej zmienności (CV > 200\%)}
	
	Cztery zmienne wykazują niezwykle wysoką zmienność, co wskazuje na ekstremalne różnice między krajami:
	
	\begin{itemize}
		\item \textbf{Population} (CV = 478,40\%) --- Liczba ludności kraju wykazuje ogromny rozrzut od 34 do 1,293,859,294 osób. Ta ekstrema zmienność odzwierciedla fundamentalnie różne rozmiary populacji krajów w zbiorze danych.
		
		\item \textbf{Measles} (CV = 473,93\%) --- Liczba zarejestrowanych przypadków odry charakteryzuje się ekstremalnymi wartościami, wahając się od 0 do 212,183 przypadków. Takie rozprzestrzenie wskazuje na ogromne różnice w rozpowszechnieniu tej choroby zakaźnej.
		
		\item \textbf{infant deaths} (CV = 389,15\%) --- Liczba zgonów niemowląt (min: 0, max: 1,800) wykazuje znaczące dysproporcje między krajami, odzwierciedlające różnice w dostępie do opieki zdrowotnej i poziomie rozwoju społeczno-gospodarczego.
		
		\item \textbf{under-five deaths} (CV = 381,69\%) --- Liczba zgonów dzieci poniżej 5 lat wykazuje podobnie wysoką zmienność, wskazując na ściśle powiązane czynniki zdrowotne z wcześniejszą zmienną.
	\end{itemize}
	
	\subsection{Zmienne o wysokiej zmienności (100\% < CV < 200\%)}
	
	Trzy zmienne wykazują wysoką zmienność w przedziale 100--200\%:
	
	\begin{itemize}
		\item \textbf{HIV/AIDS} (CV = 291,47\%) --- Rozpowszechnienie wirusa HIV wykazuje zmienność od 0,10 do 50,60, z największym rozprzestrzenieniem w krajach afrykańskich.
		
		\item \textbf{percentage expenditure} (CV = 269,27\%) --- Wydatki na opiekę zdrowotną jako procent PKB wahają się między 0,00 a 19,479,91, odzwierciedlając ogromne różnice w priorytetach budżetowych krajów.
		
		\item \textbf{GDP} (CV = 190,70\%) --- Produkt krajowy brutto krajów zawiera się w przedziale od 1,68 do 119,172,74, co wskazuje na znaczne różnice ekonomiczne między krajami.
	\end{itemize}
	
	\subsection{Zmienne o umiarkowanej zmienności (50\% < CV < 100\%)}
	
	Osiem zmiennych charakteryzuje się umiarkowaną zmiennością:
	
	\begin{itemize}
		\item \textbf{Wskaźniki niedowagi} --- Zmienne \textit{thinness 5-9 years} (CV = 92,58\%) i \textit{thinness 1-19 years} (CV = 91,33\%) odzwierciedlają znaczące różnice w zdrowotnym stanie odżywienia dzieci i młodzieży między krajami.
		
		\item \textbf{Alcohol} (CV = 88,04\%) --- Spożycie alkoholu na mieszkańca wykazuje znaczną zmienność między krajami, uwarunkowaną czynnikami kulturowymi i ekonomicznymi.
		
		\item \textbf{Adult Mortality} (CV = 75,42\%) --- Śmiertelność dorosłych standaryzowana na liczbę mieszkańców wykazuje umiarkowaną zmienność, odzwierciedlającą różne warunki zdrowotne między krajami.
		
		\item \textbf{BMI} (CV = 52,31\%) --- Średni indeks masy ciała populacji zawiera się w przedziale 1,00 do 87,30, wskazując na znaczące różnice w stanie odżywienia populacji.
	\end{itemize}
	
	\subsection{Zmienne o niskiej zmienności (CV < 50\%)}
	
	Pozostałe zmienne (Hepatitis B, Polio, Diphtheria, Total expenditure, Income composition, Schooling) wykazują relatywnie niską zmienność (CV < 50\%), co wskazuje na bardziej skoncentrowane i stabilne wartości wokół średniej.
	
	\subsection{Charakterystyka zmiennej docelowej}
	
	Zmienną docelową jest \textbf{Life expectancy} (oczekiwana długość życia), która wykazuje bardzo \textbf{stabilną zmienność}:
	
	\begin{table}[h]
		\centering
		\caption{Statystyki opisowe zmiennej docelowej}
		\label{tab:life_expectancy_stats}
		\begin{tabular}{|l|c|}
			\hline
			\textbf{Statystyka} & \textbf{Wartość} \\
			\hline
			Minimum & 36,30 lat \\
			Maksimum & 89,00 lat \\
			Średnia & 69,22 lat \\
			Mediana & 72,10 lat \\
			Odchylenie standardowe & 9,52 lat \\
			Współczynnik zmienności (CV\%) & 13,76\% \\
			\hline
		\end{tabular}
	\end{table}
	
	Niska zmienność zmiennej docelowej (CV = 13,76\%) oznacza, że oczekiwana długość życia w analizowanym zbiorze jest stosunkowo jednorodna i skoncentrowana wokół średniej. Jest to pozytywna cecha dla celów modelowania predykcyjnego.
	
	\section{Braki danych- stopień wypełnienia}
	
	\subsection{Analiza brakujących danych}
	
	Zmienne z najwyższym odsetkiem brakujących obserwacji to:
	
	\begin{table}[h]
		\centering
		\caption{Zmienne z największym udziałem brakujących danych}
		\label{tab:missing_data}
		\begin{tabular}{|l|c|c|}
			\hline
			\textbf{Zmienna} & \textbf{Liczba brakujących} & \textbf{Procent} \\
			\hline
			Population & 652 & 22,19\% \\
			Hepatitis B & 553 & 18,82\% \\
			GDP & 448 & 15,25\% \\
			Total expenditure & 226 & 7,69\% \\
			Alcohol & 194 & 6,60\% \\
			Income composition of resources & 167 & 5,68\% \\
			Schooling & 163 & 5,55\% \\
			\hline
		\end{tabular}
	\end{table}
	
	Siedem pozostałych zmiennych z brakującymi danymi (BMI, thinness, Polio, Diphtheria, Life expectancy, Adult Mortality) zawiera mniej niż 2\% brakujących wartości.
	
	\subsection{Kompletne obserwacje}
	
	Osiem zmiennych (36,36\% wszystkich zmiennych) zawiera pełne dane bez brakujących wartości:
	Country, Year, Status, infant deaths, percentage expenditure, Measles, under-five deaths, HIV/AIDS.
	
	\section{Rozkłady zmiennych i normalność rozkładu}
	
	\subsection{Histogram zmiennej docelowej}
	
	Histogram Life expectancy ujawnia rozkład zbliżony do normalnego, z dominantą wokół 72 lat. Średnia wartość (zaznaczona czerwoną linią przerywaną) pokrywa się w przybliżeniu z medianą, co wskazuje na symetryczność rozkładu. Obserwacje rozprzestrzeniają się w zakresie 36.30--89.00 lat, z rzadkimi przypadkami na krańcach.
	
	\subsection{Normalność rozkładu -- Q-Q plot}
	
	Q-Q plot (Quantile-Quantile plot) porównuje kwantyle obserwowanych danych z kwantylami rozkładu normalnego. Punkty leżą blisko linii diagonalnej, szczególnie w środkowej części rozkładu, co potwierdza \textbf{normalność rozkładu zmiennej docelowej}. Lekkie odchylenia obserwowane na krańcach rozkładu (zarówno na górze jak i na dole) są typowe dla rzeczywistych danych i nie stanowią naruszenia założenia normalności.
	
	\subsection{Box plot -- zmienne z największą zmiennością}
	
	Box plot przedstawia sześć zmiennych o największej zmienności (z wyłączeniem Population):
	
	\begin{itemize}
		\item \textbf{Zmienne z ekstremalnymi outlierami}: GDP, Measles i percentage expenditure wykazują znaczące wartości odstające (outliers) powyżej górnego wąsa. Obserwacje te są konsekwencją ekstremalnej zmienności (CV > 200\%) tych zmiennych.
		
		\item \textbf{Zmienne ze skompaktowanym rozkładem}: BMI, Diphtheria i Under-five deaths wykazują bardziej zwarte rozkłady z mniejszą liczbą wartości odstających.
		
		\item \textbf{Asymetria rozkładów}: Mediana (linia w środku pudełka) niekoniecznie dzieli pudełko na równe części, co wskazuje na asymetryczne rozkłady w niektórych zmiennych.
	\end{itemize}
	

	\begin{figure}[H]
		\centering
		\includegraphics[width=0.95\textwidth]{2.png}
		\caption{Analiza rozkładu zmiennych kluczowych: (a) histogram Life expectancy z zaznaczoną średnią, (b) Q-Q plot dla oceny normalności rozkładu, (c) box plot sześciu zmiennych o największej zmienności.}
		\label{fig:distribution_analysis}
	\end{figure}
	
	\section{Hipoteza badawcza}
	
	Oczekiwana długość życia w poszczególnych krajach jest silnie determinowana przez wskaźniki ekonomiczne (GDP), dostęp do opieki medycznej, oraz warunki sanitarne i edukacyjne. Zmienne o najsilniejszej korelacji będą stanowić predyktory modelu regresji.
	
	\section{Wnioski}
	
	Przeprowadzona analiza eksploracyjna danych WHO dotyczących oczekiwanej długości życia pozwoliła na identyfikację kluczowych czynników wpływających na zdrowie populacji. Najsilniejsze zależności ze zmienną docelową wykazały zmienne społeczno-ekonomiczne, w szczególności poziom edukacji oraz skład dochodów, co potwierdza ich fundamentalną rolę w kształtowaniu długości życia.
	
	Istotny wpływ mają również czynniki zdrowotne, takie jak śmiertelność dorosłych, rozpowszechnienie HIV/AIDS oraz poziom wyszczepienia, które w sposób bezpośredni odzwierciedlają jakość systemu opieki zdrowotnej. Analiza zmienności ujawniła znaczne dysproporcje między krajami, zwłaszcza w zakresie zmiennych demograficznych i ekonomicznych, przy jednocześnie stabilnym rozkładzie zmiennej docelowej.
	
	Uzyskane wyniki stanowią solidną podstawę do dalszego modelowania predykcyjnego oraz budowy modeli regresyjnych w kolejnych etapach projektu.
	
\end{document}